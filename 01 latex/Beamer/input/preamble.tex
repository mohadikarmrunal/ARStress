\usepackage[utf8]{inputenc}
\usetheme[block=fill,progressbar=frametitle,background=light, numbering=fraction]{metropolis}     
\usepackage[english]{babel}
\usepackage{csquotes}       
\usepackage[T1]{fontenc}        
\usepackage{booktabs}
\usepackage{pgfgantt}
\usepackage{pifont}
\usepackage{adfbullets}
\usepackage{enumitem}
\usepackage{amsmath,amsthm}   
\usepackage{tikz}
\usepackage{lipsum}
\usepackage{amssymb}
\usepackage{amsfonts}
\usepackage{mathrsfs}   
\usepackage{graphicx}
\usepackage{adjustbox}
\usepackage{varioref}
\usepackage{probsoln}
\usepackage{setspace}
\usepackage{comment}

\usepackage{tum_colors}

\setbeamertemplate{itemize items}[circle]
\setbeamertemplate{enumerate items}[default]

\DeclareMathOperator*{\argmax}{arg\,max}



\newcommand{\CL}{\ensuremath{{CL}}}
\newcommand{\NP}{\ensuremath{{0}}}
\newcommand{\BIF}{\ensuremath{{BIF}}}
\newcommand{\Bank}{\ensuremath{{Bank}}} 
\newcommand{\US}{\ensuremath{{U}}}
\newcommand{\e}[1]{{\mathbb E}\left[ #1 \right]}
\newcommand{\p}[1]{{\mathbb P}\left[ #1 \right]}
\newcommand{\iUS}{\mathbbm 1_{S^\US}}
\newcommand{\iS}{\mathbbm 1_{S}}
\newcommand{\iCL}{\mathbbm 1_{S^\CL}}
\newcommand{\iBIF}{\mathbbm 1_{S^\BIF}}
\newcommand{\rf}{\ensuremath{r}}

\makeatletter

\setlength{\metropolis@titleseparator@linewidth}{2pt}
\setlength{\metropolis@progressonsectionpage@linewidth}{2pt}
\setlength{\metropolis@progressinheadfoot@linewidth}{2pt}
\newcommand*{\rom}[1]{\expandafter\@slowromancap\romannumeral #1@}

\makeatother

\setbeamertemplate{frame footer}{\insertshortauthor}
\setbeamerfont{page number in head/foot}{size=\tiny}
\setbeamercolor{footline}{fg=gray}

\setbeamercolor{frametitle}{bg=white, fg=black}

\usepackage{booktabs}

\newtheorem{proposition}{Proposition}
\usepackage{subfigure,placeins,subcaption}
\usepackage{fontspec}
\setsansfont{Ubuntu}

\usepackage{natbib}
\let\realcitep\citep
\renewcommand*{\citep}[1]{{\footnotesize\realcitep{#1}}}


\usepackage{rotating}
\usepackage{siunitx}
\usepackage{dcolumn}
\newcolumntype{d}[1]{D{.}{.}{#1}}
\newcommand{\muc}[2]{\multicolumn{1}{C{#1}}{#2}}
\newcommand{\mul}[2]{\multicolumn{1}{L{#1}}{#2}}
\newcommand{\mucThree}[2]{\multicolumn{3}{C{#1}}{#2}}
\newcommand{\mulTen}[2]{\multicolumn{10}{L{#1}}{#2}}

\newcolumntype{C}[1]{>{\centering\arraybackslash}p{#1}} 
\newcolumntype{L}[1]{>{\raggedright\arraybackslash}p{#1}} 
\newcolumntype{R}[1]{>{\raggedleft\arraybackslash}p{#1}} 
\newlength\interColSepSmall
\newlength\interColSepLarge
\setlength{\interColSepSmall}{0.2em}
\setlength{\interColSepLarge}{0.4em}
\usepackage{bbm}

\usepackage{multirow,setspace}
\usepackage{minibox}

\setlength{\itemindent}{-0.4em}
\usepackage{pst-all}
\usepackage{pst-plot}


\newcommand{\positionLogo}{
    \begin{tikzpicture}[remember picture,overlay]
        \node[anchor=north east, inner sep=5pt] at (current page.north east) {\includegraphics[width=1cm]{01 latex/Beamer/input/TUM_Logo_blau_rgb_p.png}};
    \end{tikzpicture}
}

%\addtobeamertemplate{frametitle}{}{\positionLogo} % Position logo on each frame

\setbeamertemplate{itemize item}{\scriptsize\raise1.25pt\hbox{\textbullet}}
\setbeamertemplate{itemize subitem}{\tiny\raise1.5pt\hbox{\textbullet}}
\setbeamertemplate{itemize subsubitem}{\tiny\raise1.5pt\hbox{\textbullet}}

\usepackage{adjustbox} % Add this line in your preamble if not already included
%\newcommand{\setmyheader}{
%    \setbeamertemplate{headline}{
%        \begin{beamercolorbox}[wd=\paperwidth, ht=7.5ex, dp=1.125ex]{bg=white, fg=white}
%        \hspace*{\dimexpr\paperwidth-0.3cm} % Move to the right
%            \vspace*{-0.4cm} % Move down
%            \includegraphics[width=1cm]{TUM_Logo_blau_rgb_p.png} % Adjust width as needed
%            % Add any additional content here
%        \end{beamercolorbox}
%    }
%}

\newcommand{\HypothesisArrow}[1]{%
    \raisebox{4ex}{%
        \begin{tikzpicture}[line/.style={draw, black}]
            \draw[line, {Latex}-{Latex}] (-1,2.5) -- (1,2.5) 
                node[midway, below] {\small(Hypothesis #1)};
        \end{tikzpicture}
    }%
}

